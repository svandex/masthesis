\chapter{部分代码文件}
\section{离子电流双干扰确定算法}
\label{cd:dintf}
\lstset{
	basicstyle=\ttfamily\scriptsize,
	breaklines=true}
	
\begin{lstlisting}[frame=single,language=matlab]
function [fA,fB]=fluccalc(y,dA,dB)
% y	data of ion of a single cycle
% dA 	the offset of interfere A
% db 	the offset of interfere B
% fA 	struct for start point and stop point of interfere A
% fB 	struct for start point and stop point of interfere B
switch nargin
    case 1
        dA = 0; dB = 0;
    case 2
        dB = 0;
    otherwise
end

y_a = dgtflt(diff(y)); %digital filter function

for num = 1:length(y_a)
    if y_a(num)>0.1
        flucAStart = num+dA;
        break;
    end
end

for num = (flucAStart+200):-1:flucAStart
    if y_a(num)>0.1
        flucAEnd = num;
        break;
    end
end

for num = (flucAStart+200):length(y)
    if y_a(num) >0.1
        flucBStart = num+dB;
        break;
    end
end

flucBEnd = flucBStart+200;

fA = struct('fAs',flucAStart,'fAe',flucAEnd);
fB = struct('fBs',flucBStart,'fBe',flucBEnd);
\end{lstlisting}
\section{双高斯曲线拟合算法}
\label{cd:dgauss}
\lstset{
	basicstyle=\ttfamily\scriptsize,
	breaklines=true}
	
\begin{lstlisting}[frame=single,language=matlab]
function [lambda,A,peakHeights,ry] = iongaussfit(y,dg1,dg2)
% y	ion data sequence from 0 to 720 degree.
% lambda	a matrix of 2x2, each row is responsive to a gaussian curve
% dg1	start degree of fitting
% dg2	end degree of fitting
% ry	real ion data
% A	gaussian data matrix
% peakHeights peak height of each gaussian

global numpeaks ns pos1 pos2 ca;
if dg1>=dg2||dg1>=360
    disp('dg1 arg fault!');
    return;
end
ca = dg2-dg1;
dg1num = round(length(y)*dg1/720);
dg2num = round(length(y)*dg2/720);
ry = y(dg1num:dg2num);

ns = 1:length(ry);
numpeaks = 2;
[~,tempos] = max(ry);

%gaussian parameter initial setting
pos1 = round(length(ry)*(360-dg1)/(dg2-dg1)); % pos1 = ca50
pos2 = tempos; % pos2 = the maximum value of ion
slambda = [pos1,1200,pos2,100];

options = optimset('TolX',.00001);
plambda = fminsearch(@fitgaussian,slambda,options,ry);

A(:,1) = gaussian(ns,plambda(1),plambda(2));
A(:,2) = gaussian(ns,plambda(3),plambda(4));

ppeakHeights = abs(A\ry);

% test(A,ppeakHeights,ry,dg1,dg2);

[~,phm] = min([plambda(2),plambda(4)]);
peakHeights = zeros(2,1);
peakHeights(1) = ppeakHeights(3-phm);
peakHeights(2) = ppeakHeights(phm);

lambda(1) = (dg2-dg1)*plambda(2*(3-phm)-1)/length(ry)+dg1;
lambda(2) = (dg2-dg1)*plambda(2*(3-phm))/length(ry);
lambda(3) = (dg2-dg1)*plambda(2*phm-1)/length(ry)+dg1;
lambda(4) = (dg2-dg1)*plambda(2*phm)/length(ry);
end

function err = fitgaussian(lambda,y)
    global ns;
    A(:,1) = gaussian(ns,lambda(1),lambda(2));
    A(:,2) = gaussian(ns,lambda(3),lambda(4));

    peakheights = abs(A\y);
    z = A*peakheights;
    err = norm(z-y);
end

function g = gaussian(x,pos,wid)
    g = exp(-((x-pos)./(0.6005615.*wid)) .^2);
end

function test(A,peakHeights,y,dg1,dg2)
% TEST SCRIPTS
global numpeaks;
mmodel = A*peakHeights;
rge = (1:length(y))*(dg2-dg1)/length(y)+dg1;
figure('unit','centimeters','position',[2 10 14.6*0.8 8]);hold on;
plot(rge,mmodel,'r');plot(rge,dgtflt(y,4),'b');
for j=1:numpeaks
    plot(rge,A(:,j)*peakHeights(j));
end
legend('sim model','real','gauss1','gauss2','location','northwest');
xlabel('crank angle(CA)');ylabel('ion voltage(V)');
ylimvalue = get(gca,'ylim');
ylim([0 ylimvalue(2)]);
% xlim([dg1 dg2]);
set(gca,'fontsize',8);

end
\end{lstlisting}