\chapter{部分代码文件}
\section{离子电流双干扰确定算法}
\label{cd:dintf}
\lstset{
	basicstyle=\ttfamily\scriptsize,
	breaklines=true}
	
\begin{lstlisting}[frame=single,language=matlab]
function [fA,fB]=fluccalc(y,dA,dB)
% y	data of ion of a single cycle
% dA 	the offset of interfere A
% db 	the offset of interfere B
% fA 	struct for start point and stop point of interfere A
% fB 	struct for start point and stop point of interfere B
switch nargin
    case 1
        dA = 0; dB = 0;
    case 2
        dB = 0;
    otherwise
end

y_a = dgtflt(diff(y));

for num = 1:length(y_a)
    if y_a(num)>0.1
        flucAStart = num+dA;
        break;
    end
end

for num = (flucAStart+200):-1:flucAStart
    if y_a(num)>0.1
        flucAEnd = num;
        break;
    end
end

for num = (flucAStart+200):length(y)
    if y_a(num) >0.1
        flucBStart = num+dB;
        break;
    end
end

flucBEnd = flucBStart+200;

fA = struct('fAs',flucAStart,'fAe',flucAEnd);
fB = struct('fBs',flucBStart,'fBe',flucBEnd);
\end{lstlisting}
\section{双高斯曲线拟合算法}
\label{cd:dgauss}
\lstset{
	basicstyle=\ttfamily\scriptsize,
	breaklines=true}
	
\begin{lstlisting}[frame=single,language=matlab]
function [lambda,A,peakHeights] = iongaussfit(y)
	% y         ion data sequence from 270 to 450 degree.
	% lambda    a matrix of 2x2, each row is responsive to a gaussian curve
	global numpeaks ns pos1 pos2;
	ns = 1:length(y);
	numpeaks = 2;
	[~,tempos] = max(y);

	%gaussian parameter initial setting
	pos1 = round(length(y)/2); % pos1 = 360 dgr
	pos2 = tempos; % pos2 = the maximum value of ion
	slambda = [pos1,1200,pos2,100];

	options = optimset('TolX',.00001,'Display','iter' );
	lambda = fminsearch(@fitgaussian,slambda,options,y);

	A(:,1) = gaussian(ns,lambda(1),lambda(2));
	A(:,2) = gaussian(ns,lambda(3),lambda(4));

	peakHeights = abs(A\y);

	test(A,peakHeights,y); %plot model and each gaussian curve
end

function err = fitgaussian(lambda,y)
	global ns;
	A(:,1) = gaussian(ns,lambda(1),lambda(2));
	A(:,2) = gaussian(ns,lambda(3),lambda(4));

	peakheights = abs(A\y);
	z = A*peakheights;
	err = norm(z-y);
end

function g = gaussian(x,pos,wid)
	g = exp(-((x-pos)./(0.6005615.*wid)) .^2);
end

function test(A,peakHeights,y)
	% TEST SCRIPTS
	global numpeaks;
	mmodel = A*peakHeights;
	rge = (1:length(y))*180/length(y)+270;
	
	figure('unit','centimeters','position',[2 2 14.6*0.8 8]);hold on;
	plot(rge,mmodel,'r');plot(rge,dgtflt(y,4),'b');
	
	for j=1:numpeaks;
		plot(rge,peakHeights(j)*A(:,j));
	end
	
	legend('sim model','real','gauss1','gauss2');
	xlabel('crank angle(CA)');ylabel('ion voltage(V)');
	ylim([0 1]);xlim([270 450]);
	set(gca,'fontsize',8,'xtick',270:30:450);

end
\end{lstlisting}